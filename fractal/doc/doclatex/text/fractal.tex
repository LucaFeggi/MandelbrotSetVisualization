\chapter{todo/revision history}
\begin{verbatim}
TODO
functionality
    numeric calculus 
        range/interval/window size
        precision abcisses, ordenate axis/resolution of image
        maximum module
            coordinates c where maximum module
        number of elements

model
    mandelbrot set sequence
    beloging to mandelbrot set  


    fractal boundary
    module of a complex number

    data structure
        big numbers
            many significant digits
        
        cartesian coordenates
            real and imaginary parts of mandelbrot set's elements

view
    cartesian representation


REVISION HISTORY


16/01/2024   15:15-17:00    fractal theory
                            complex number library
                            printing module-library
                            mandelbrot algorithm code

18/01/2024   16:00-19:00    mandelbrot theory
                            mandelbrot algorithm
                            traverse algorithm

20/01/2024   10:00-11:00    data print
                            precision
                            isMandelbrotElement()
\end{verbatim}



\chapter{hologram}
\chapter{fractal}
\section{theory}
\noindent \url{https://en.wikipedia.org/wiki/Fractal}\\
\\
a fractal is a geometrical shape\\
a curve\\
\\
a plane\\
Sierpiński carpet\\
the technique of subdividing a shpe into smaller copies of itself
and removing one or more copies\\
\url{https://en.wikipedia.org/wiki/Sierpi%C5%84ski_carpet}
\\
a three dimensions, cubes\\
Menger sponge\\
\url{https://en.wikipedia.org/wiki/Menger_sponge}\\

a function/equation\\
increase presision in\\
domain\\
different, and not necessarily more precise only,\\
image\\
\\
infinitely increase in precision\\
\\
because sequence defined recursively\\

\chapter{mandelbrot set}
\url{https://en.wikipedia.org/wiki/Mandelbrot_set}
\url{https://ics.uci.edu/~eppstein/junkyard/mand-area.html}
\url{https://math.stackexchange.com/questions/1134054/proof-of-x-intersection-of-the-mandelbrot-set}
\url{https://complex-analysis.com/content/mandelbrot_set.html}
\\
\noindent buddhabrot representation\\
\url{https://www.kaggle.com/code/wgunderwood/the-mandelbrot-set}\\
\begin{verbatim}
    definition
    ----------
        Mandelbrot set 'M'
            two-dimensional set defined in the complex-number plane

            compact set
                since it is closed (no punctures)
                and bound subset (no missing points)
                    contained in the closed disk of radius 2 around an origin (z0=0+0j)

            element/point 'c' is in 'M'
                if
                    module '|z|' complex number 'z' in sequence f(z) = z^2 + c, defined recursively by function
                    absolute value of z_{n}, |z_{n}| <= 2, for all n >= 0

                otherwise 
                    absolute value exceeds 2, the sequence will escape to inifinity therefore not in 'M'

            sequence
            ----------
                origin      o = or + oi·j
                            z0 = zr0 + zi0·j
                            z0 = o
                            zr0 = or
                            zi0 = oi

                distance    d0 = dr0 + di0·j
                               = z0 - o
                               = zr0 + zi0·j - or - oi·j
                               = zr0-or + (zi0-oi)·j
                               = or-or + (oi-oi)·j
                               = 0 + 0·j
    
                candidate   c = a + b·i 

                |z0|+2
                f(z) = z^2 + c is <= |z0|+2

                complex number z0 =             0 + 0i                          ;   |z0|   =     (a^2 + b^2)^1/2    <= |z0| + 2
                complex number z1 = f(z0)       = z0^2 + c  = (a+bi)^2 + c      ;   |z1|   =     |c|                =  (a^2 + b^2)^1/2
                complex number z2 = f(z1)       = z1^2 + c  = c^2 + c           ;   |z2|   =     |c^2 + c|          = 
                complex number z3 = f(z2)       = z2^2 + c  = (c^2 + c)^2 + c

                complex number |z0|   =     (0^2 + 0^2i)^1/2
                complex number |z1|   = 
                complex number |z2|   = 
                complex number |z3|   = 
                
                no diverge to infinity

    axis intersection
    ----------
        the intersection of 'M' with the real axis is the interval [-2, 1/4]
        abscissa    x-axis  real part
        ordinate    y-axis  imaginary part

    fractal boundary
        definition 
            boundary of the Mandelbrot set 'M' is a fractal curve.
            boundary complex numbers which magnitude is 2, threshold must be at least 2.

            complex number z = -2 + 0i, z = -2
            has the largest magnitue with in the mandelbrot set








JULIA SET
        if c is held constant and the inital value of 'z' is varied instead, the corresponding Julia set for the point c is obtained.





COMPLEX NUMBERS
        Function module/magnitude |z| of a complex number 'z', z=a+bj
        https://en.wikipedia.org/wiki/Complex_number
            z = x + y·j
            |z| = sqrt(x^2 + y^2)







COMPUTING
    element 'c' in 'M'
        maximum number of iterations n=500?
        all '|z|' <= 2

    element 'c' not in 'M'
        one '|z|' > 2

        infinity inf=10^8?

    ordenate bounds
    
    fractal dimensions
        cartisian vertix coordinates 
            (-2.0,2.0), (0.25, 2.0)
            (-2.0,-2.0), (0.25,-2.0)




COLORING
    only boundary, in black color
        2 >= |z| >= 1.995
    
    coordinate
        treating the real and imaginary parts of 'c' as image coordinates

    pixel
        pixels may be colored according to how soon the sequence
            |f_{c}(z)|, |f_{c}(f_{c}(z))|, |f_{c}(f_{c}(f_{c}(z)))|
            crosses an arbitrarily chosen threshold
                the threshold must be at least 2, as -2 is the complex number 
                    with the largest magnitude within the set,
                but otherwise the threshold is arbitrary.

                mandelbrot subset by threshold lower than 2???

    soon
        number of iteration

    distance
        difference |2-|z_{i}||, |z_{i}| <= 2, |z_{i+1}| > 2






    color gradient
    https://en.wikipedia.org/wiki/Color_gradient
        axial gradient
            axial gradient or, also called, linear color gradient
            segment defined by two points
            on color for each point of the segment

        color space
            2d
                plane (r in [0,255], g in [0,255])

            3d
                rgba (0-255, 0-255, 0-255, 0-255)
                2d rgb profiles
                https://en.wikipedia.org/wiki/Color_gradient#/media/File:Gnuplot_HSV_gradient.png
                https://en.wikipedia.org/wiki/Color_gradient#/media/File:Matlab_gradient.png

                3d rgb profiles
                https://en.wikipedia.org/wiki/Color_gradient#/media/File:0_3d_60_75_v.png

HISTORY
 first
    Benoit Mandelbrot
    mathematician
    coined word 'fractal'
    wrote influencial book 'The Fractal Geometry of Nature', in 1982.
    first mandelbrot set image in the cover of 'Scientific American', 1985.
since then
\end{verbatim}

\chapter{image}
\section{file format}
\begin{verbatim}
TYPE OF IMAGE FORMAT
----------------------------------------

raw images
    camara lenses capture raw information

png images
    a png is a raster: pixel-based image format
    stores pixels in a raster format
    one color channel
        monochrome or palette index
    three color channel
        rgb
    four color channel
        rgba, rgb with an alpha channel for transparency

jpg images
    no alpha channel support
    color information retained in rgb
    compression by discarding data

svg
    not a vector image like svg

raw pixel images dataset
    http://yann.lecun.com/exdb/mnist/


" .:-=+*#\%@"
" .°*oO#@"
" _.,-=+:;cba!?0123456789\$W#@"


\end{verbatim}

\section{greyscale}
\begin{verbatim}
MATRIX OF GREYSCALE MODEL
https://en.wikipedia.org/wiki/Grayscale
----------------------------------------

    values in each pixel is a single sample representing only
the amount of light, intensity of light, luminance, brightness.

    numerical representation of a greyscale
commonly stored in 8 bits/pixel per sampled pixel

converting color to greyscale
    no alpha component

    luminance Y_{lineal}
    Y_{lineal} = 0.2126·R_{linear} + 0.7152·G_{linear} + 0.0722·B_{linear}
    Y_{lineal} belongs to [0,1]



\end{verbatim}

\section{rgb}
\begin{verbatim}
RGB COLOR MODEL
https://en.wikipedia.org/wiki/RGB_color_model
----------------------------------------

truecolor image
    linear colorspace
    human perception

additive model
    addition or combination of primary colors red, green, blue in different intensites
    zero intensity of each component gives black
    maximum intensity of each component gives white

three color channels 8 bits per channel, values 0-255 
     24 bits/pixel
     (2^8)^3 = 16777216 colors
\end{verbatim}



\chapter{integer}
\noindent gnu libc\\
integers\\
\url{https://www.gnu.org/software/libc/manual/html\_node/Integers.html}\\


% DOCUMENT

%\documentclass[a4paper, 12pt, spanish, fleqn]{report} % fleqn test align formulas left
\documentclass[a4paper, 12pt, fleqn]{report} % fleqn test align formulas left
\usepackage{geometry} % Required for adjusting page dimensions and margins
\geometry{
	paper=a4paper, % Paper size, change to letterpaper for US letter size
    paperheight=29.7cm,
    paperwidth=21cm,
    includehead,
    nomarginpar,
    textwidth=15cm,
	top=2.5cm, % Top margin
	bottom=3cm, % Bottom margin
	left=3cm, % Left margin
	right=3cm, % Right margin
	headheight=10mm, % Header height
	footskip=15mm, % Space from the bottom margin to the baseline of the footer
	headsep=7.5mm, % Space from the top margin to the baseline of the header
	%showframe, % Uncomment to show how the type block is set on the page
}


% LANGUAGES

\usepackage[english, spanish]{babel} % English language hyphenation
%\usepackage{spanish} % Spanish language hyphenation, date
%\usepackage{babel-spanish} % Spanish language hyphenation, date

% FONTS

\usepackage[utf8]{inputenc}
%\usepackage[latin1]{inputenc}
\usepackage[T1]{fontenc}

%\usepackage{courier}
%\usepackage[nottoc]{tocbibind} %list figures, tables in content


    % palentino font

\usepackage{mathpazo} % Use the Palatino font
\usepackage[protrusion=true, expansion=true]{microtype} % Better typography
\linespread{1.05} % Increase line spacing slightly; Palatino benefits from a slight increase by default




% FONETIC SYMBOLS
\usepackage{tipa}

% PART

% removing page numbers in part pages, only for book,report template
\usepackage{etoolbox}
\patchcmd{\part}{\thispagestyle{plain}}{\thispagestyle{empty}}{}{}


% spaces in the table of contents

\usepackage{tocloft}

    % title in table of content

\setlength{\cftbeforetoctitleskip}{-3em}
\renewcommand{\cfttoctitlefont}{\hfill\LARGE\bfseries\scshape} % added
\renewcommand{\cftaftertoctitle}{\hfill\vspace{15pt}} % centering and vertical space after of toc title 
%\renewcommand{\cftaftertoctitle}{\bigskip}
    

% center part title and number size (zero size to remove it?)

    % title in table of content's parts
\renewcommand{\cftpartfont}{\hfil\LARGE\bfseries\scshape}


\renewcommand{\cftpartleader}{\hfil}
\setlength{\cftpartnumwidth}{5pt}

% vertical line space
    % before toc titles
\setlength{\cftbeforepartskip}{1.5em}
\setlength{\cftbeforechapskip}{0.5em}

    % after toc titles
\renewcommand\cftpartafterpnum{\vskip5pt}
\renewcommand\cftchapafterpnum{\vskip2pt}
\renewcommand\cftsecafterpnum{\vskip2pt}
\renewcommand\cftsubsecafterpnum{\vskip2pt}

%  horizontal line space 

    % after toc titles
\renewcommand\cftchapafterpnum{\hspace*{1.0em}}
\renewcommand\cftsecafterpnum{\hspace*{2.0em}}
\renewcommand\cftsubsecafterpnum{\hspace*{3.0em}}

% indent
\cftsetindents{chapter}{1.25cm}{1.25cm}
\cftsetindents{section}{2.50cm}{1.00cm}
\cftsetindents{subsection}{3.75cm}{1.25cm}







% TIME
\usepackage{datetime}


% HYPERREF

\usepackage{hyperref}
\hypersetup{
  %hidelinks,
  colorlinks   = true, % Colours links instead of ugly boxes
  urlcolor     = black, % Colour for external hyperlinks
  linkcolor    = black, % Colour of internal links
  citecolor   = black % Colour of citations
}
\usepackage{url}



% EPIGRAPHS

\usepackage{epigraph}

\setlength\epigraphwidth{.8\textwidth}
\setlength\epigraphrule{0pt}





% IMAGES

\usepackage{graphicx}
\graphicspath{{image/}}



% MATH

\usepackage{mathtools}
\usepackage{amsmath}
\usepackage{amsfonts}
    % mthbb{} 


% UNIT

\usepackage{siunitx}
\usepackage{steinmetz}




% HEADER FOOTER
\usepackage{fancyhdr}
\usepackage{lastpage}

\renewcommand{\headrulewidth}{2pt}
%\renewcommand{\headrulewidth}{0.1pt}
%\renewcommand{\headrulewidth}{0 pt}

\renewcommand{\footrulewidth}{1pt}
%\renewcommand{\footrulewidth}{0.1pt}
%\renewcommand{\footrulewidth}{0 pt}


% COLOURS

\usepackage[dvipsnames]{color, xcolor, colortbl}
%\definecolor{name}{system}{definition}
\definecolor{Gray}{gray}{0.9}
\definecolor{LightCyan}{rgb}{0.88,1,1}
\definecolor{Crane}{rgb}{52,109,0}
\definecolor{Amber}{rgb}{1.0,0.75,0}
\definecolor{Amberb}{rgb}{1.0,0.49,0}
\definecolor{mypink1}{rgb}{0.858, 0.188, 0.478}
\definecolor{mypink2}{RGB}{219, 48, 122}
\definecolor{mypink3}{cmyk}{0, 0.7808, 0.4429, 0.1412}
\definecolor{mygray}{gray}{0.6}



% TABLE

% TABLE MULTICOLUMN

\usepackage{multicol}
\usepackage{multirow}

% TABLE CAPTION

\usepackage{caption} 
% package for listings titles
% package for tables captions
%  caption same with as tabular envioroment
%\usepackage[tableposition=top]{caption} % table caption on top
\captionsetup{
                labelfont=bf,
                justification=raggedright,
                singlelinecheck=false
            }

\usepackage{subcaption}

\usepackage{floatrow}
%  Table caption on top
\floatsetup[table]{capposition=top}
%  Table caption on bottom 
%\floatsetup[table]{capposition=below}

% numbering tables only in chapters, in non-chapters table numbering
% + starts at zero.
\usepackage{chngcntr}
\counterwithin{table}{chapter}

% TABLE RULE

\usepackage{booktabs} % Required for better horizontal rules in tables

% TABLE LONG TABLE

\usepackage{longtable}
\setcounter{LTchunksize}{10}

% centered is default
%\begin{longtable}[c]{...}
%\setlength{\LTright}{0pt} % long tables aligned left
%\setlength{\LTleft}{0pt} % long tables aligned left
%\setlength{\LTright}{\fill} % long tables aligned left
%\setlength{\LTleft}{\fill} % long tables aligned left
%\LTcapwidth=\textwidth

% TABLE DASH LINES

\usepackage{arydshln} % dash lines in tables \hdashline
% package must be installed after package longtable

% TABLE COLUMN FORMAT - JUSTIFICATION FORMAT
\usepackage{array}




% CHEMSCHEMES
\usepackage{chemscheme}



% ITEMIZED,ENUMERATED LISTS
% left margins [leftmargin=5.5mm]
\usepackage{enumitem}
\newlist{myenumerate}{enumerate}{1}
\setlist[myenumerate,1]{label=\arabic*. , leftmargin=*, before=\color{orange}, font=bfseries, format=\color{red}}






% LISTINGS
% lstautogobble, removes begging whitespaces and cares about indentation
\usepackage{listings,lstautogobble}

% accents in text within listings enviroment
% goes with \lstset{inputencoding=utf8/latin1}
\usepackage{listingsutf8}

% Importing code from file
%\lstinputlisting[language=Octave]{BitXorMatrix.m}
%\lstinputlisting[language=Python]{source_filename.py}
%\lstinputlisting[language=Python, firstline=37, lastline=45]{source_filename.py}
%\lstinputlisting[language=Python, linerange={37-45,48-50}]{source_filename.py}

\lstdefinestyle{mystyle}{
    autogobble=true;
    %escapechar=\,
    %mathescape=true,
    %
    frame=single,
    %frame=lines,
    frame=tblr,
    framerule=10pt,
    rulecolor=\color{green},
    %rulesep=1pt,
    %
    tabsize=4,
    % 
    basicstyle = \fontsize{12}{13}\selectfont\ttfamily,
    columns = fullflexible,
    %
    numbers=none,
    numberstyle=\tiny\color{gray},
    %numbers=left,
    %stepnumber=1,
    %numbersep=3pt,
    %firstnumber=1,
    % 
    backgroundcolor=\color{white},
    %
    extendedchars = false,
    breaklines = true,
    breakatwhitespace=true,
    showtabs = false,
    showspaces = false,
    showstringspaces=false,
    %
    captionpos=b,
    %captionpos=t,
    % 
    % DISTANCE TO PAGE MARGIN
    aboveskip=10mm,                      % space between top text and code 
    belowskip=10mm,                     % space between bottom code and text
    abovecaptionskip=8mm,
    belowcaptionskip=0mm,
    %belowcaptionskip=.10\baselineskip,
    %
    framesep=5mm,                       % code margins to frame
    xleftmargin = 0.0\textwidth,        % code left margin to frame
    xrightmargin = 0.0\textwidth,       % code right margin to frame
    framextopmargin=0mm,
    framexbottommargin=0mm,
    framexleftmargin=0mm,
    framexrightmargin=0mm,
    %
}

\lstdefinestyle{mystyle2}{
    autogobble=true,
    %language = Java,
    %float=!h,
    %frame=tblr,
    %
    captionpos=b,
    %
    backgroundcolor = \color{white},
    %backgroundcolor = \color{pink},
    %backgroundcolor=\color{lightgray},
    %
    extendedchars = false,
    breaklines = true,
    breakatwhitespace=true,
    showtabs = false,
    showspaces = false,
    showstringspaces=false,
    %
    aboveskip=5em,
    belowskip=2em,
    abovecaptionskip=1em,
    belowcaptionskip=.10\baselineskip,
    % 
    numberstyle=\tiny\color{gray},
    %numberstyle = \tiny\color{black},
    %numberstyle = \tiny\color{red},
    tabsize=4,
    %numbers=left,
    numbers=none,
    % 
    basicstyle = \fontsize{12}{13}\selectfont\ttfamily,
    %basicstyle = \small,
    columns = fullflexible,
    % 
    keywordstyle=\color{black},
    %keywordstyle=\color{blue},
    stringstyle=\color{black},
    %stringstyle=\color{red},
    commentstyle=\color{Green}\ttfamily
    %commentstyle=\color{black},
    %commentstyle=\color{green},
    % 
}


\lstset{style=mystyle}


\colorlet{listingscolor}{black!15}

\lstnewenvironment{mylistings}
  {\lstset{language=C++,
    backgroundcolor=\color{listingscolor}, % set backgroundcolor
    basicstyle=\footnotesize,% basic font setting
    }
}
{}



\lstnewenvironment{cpptable} {
    \lstset{
        language=C++,
        basicstyle=\footnotesize,% basic font setting
    }
}
{}


% setting the font to courier
% https://tex.stackexchange.com/questions/33685/set-the-font-family-for-lstlisting
% https://tex.stackexchange.com/questions/378623/how-can-i-get-a-list-of-all-fontfamily-fontseries-combinations/378626#378626
\lstnewenvironment{revhistable} {
    \lstset{
        basicstyle=\fontfamily{ppl}\normalsize,
        %inputencoding=utf8/latin1,
        inputencoding=utf8,
        %inputencoding=latin1,
        %extendedchars=true,
        frame=none,
        %frame=tblr,
        numbers=none,
        backgroundcolor = \color{white},
        xleftmargin=0.0cm, 
        xrightmargin=0.2\textwidth,
        framexleftmargin=0.0cm,
        framexrightmargin=-1.5cm,
        aboveskip=0.0cm,
        escapeinside={(*}{*)},
        literate=
        {á}{{\'a}}1
        {à}{{\`a}}1
        {ã}{{\~a}}1
        {é}{{\'e}}1
        {ê}{{\^e}}1
        {í}{{\'i}}1
        {ó}{{\'o}}1
        {õ}{{\~o}}1
        {ú}{{\'u}}1
        {ü}{{\"u}}1
        {ç}{{\c{c}}}1
    }
}
{}


% Caption Title Name "Program"
\renewcommand{\lstlistingname}{Program} 

% Caption format
% https://latex-tutorial.com/caption-customization-latex/
%\DeclareCaptionFont{font1}{\color{black}}
%\DeclareCaptionFormat{listing1}{\rule{\dimexpr\textwidth+17pt\relax}{0.4pt}\vskip1pt#1#2#3}
%\DeclareCaptionFormat{listing2}{
%    \colorbox[cmyk]{0.43, 0.35, 0.35,0.01 }{\parbox{\textwidth}{\hspace{15pt}#1#2#3}
%    }
%  }
%\captionsetup[lstlisting]{singlelinecheck=false, margin=0mm, labelfont={font1}, textfont={font1}, font={sf, bf,footnotesize}}
%\captionsetup[lstlisting]{singlelinecheck=false, margin=0pt, labelfont={font1}, textfont={font1}, font={sf, bf,footnotesize}}
%\captionsetup[lstlisting]{format=listing2, singlelinecheck=false, margin=0pt, labelfont={font1}, textfont={font1}, font={sf, bf,footnotesize}}




% Lists

%\renewcommand*\contentsname{Table of Contents} % change Content name
%\renewcommand\listfigurename{Lista de Figuras}
%\renewcommand\listtablename{Lista de Tablas}




% FLOATS

\usepackage{float} % [H] option, non floating figures, in-text
\usepackage{newfloat} % Creating New Float Enviroments
    \DeclareFloatingEnvironment[placement={!ht},name=List]{mylist}



% DRAWING

\usepackage{pgfplots}

% TIKZ

\usepackage{tikz}
\usepackage[siunitx, RPvoltages, europeanresistors]{circuitikz}
\usetikzlibrary{arrows, shapes.geometric}


\tikzset{
  treenode/.style = {align=center, inner sep=0pt, text centered,
    font=\sffamily},
  arnn/.style = {treenode, circle, white, font=\sffamily\bfseries, draw=black,
    fill=black, text width=1.5em},% arbre rouge noir, noeud noir
  arnr/.style = {treenode, circle, red, draw=red,
    text width=1.5em, very thick},% arbre rouge noir, noeud rouge
  arnx/.style = {treenode, rectangle, draw=black,
    minimum width=0.5em, minimum height=0.5em}% arbre rouge noir, nil
}

% Cronograms

\newcounter{wavenum} % Cronogramas

% subscripts out math enviroment

\usepackage{relsize}
\def\textsubscript#1{\ensuremath{_{\mbox{\textscale{.6}{#1}}}}}

% IC Circuit

\usepackage{relsize}
\usepackage{pbox}
\newcommand{\ctikzlabel}[2]{\pbox{\textwidth}{#1\\#2}} % multiple-lines labels



% BIBLIOGRAPHY
% Numbering Bibliography

%\usepackage[superscript]{cite} % numbering in supperscript
\usepackage[superscript,biblabel]{cite} % numbering in supperscript

% inclusion in table Content

% without toc numbering
%\usepackage[nottoc, notlof,notlot]{tocbibind} 

% with toc numbering
% \usepackage[nottoc, numbib,notlof,notlot]{tocbibind} 


